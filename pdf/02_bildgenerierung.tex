\chapter{Modell}
\label{ch:model}

\section{Tensoren}
Mehrdimensionale Arrays, in NumPy ndarray (N-Dimensional Array) genannt, werden auch Tensoren genannt. Gennerell verwenden alle derzeitigen Systeme für Maschinelles Lernen Tensoren als zugrundeliegende Datenstruktur. Tensoren sind elementar für dieses Fachgebiet. \cite{chollet2017dlpython}

In seinem Kern ist ein Tensor ein Container für numerische Daten. Tensoren sind die Verallgemeinerung der Matrizen für eine beliebige Anzahl Dimensionen. Im Zusammenhang mit Tensoren werden die Dimensionen häufig Achsen gennannt. Die Anzahl der Achsen wird auch als Stufe bezeichnet. Ein Skalar ist ein Tensor nullter Stufe. Ein Vektor ist ein Tensor erster Stufe und eine Matrix ein Tensor zweiter Stufe. \cite{chollet201chollet2017dlpython7}

Der Begriff der Dimension, der für Vektoren und Matrizen eindeutig ist, wird für Tensoren in der Literatur unterschiedlich verwendet.  \cite{chollet2017dlpython} In dieser Arbeit soll die Dimension einer Achse eines Tensors die Anzahl der numerischen Werte entlang dieser Achse bezeichnen.

Bilder erscheinen als Tensoren dritter Stufe, deren Achsen der Höhe, Breite und den Farbkanälen (Rot, Grün und Blau) entsprechen. \cite{zhang2020dive}

Ein Tensor ist durch drei Schlüsselattribute definiert:
\begin{itemize}
\item Anzahl der Achsen (Stufe): Zum Beispiel hat ein Tensor dritter Stufe drei Achsen, und eine Matrix hat zwei Achsen. In Python-Bibliotheken wie NumPy oder Tensorflow wird sie auch als $ndim$ des Tensors bezeichnet.
\item Form: Dies ist ein Tupel aus Ganzzahlen, das die Dimension jeder Achse beschreibt. Beispiele sind $(3, 5)$ für die Form einer Matrix und $(3, 3, 5)$ für einen Tensor dritter Stufe. Ein Vektor hat eine Form mit einem einzelnen Element, etwa so: $(5,)$, während ein Skalar eine leere Form hat: $()$.
\item Datentyp (in Python-Bibliotheken gewöhnlich $dtype$ genannt): Dies ist der Typ der im Tensor enthaltenen Daten. Zum Beispiel könnte der Datentyp eines Tensors $float16$, $float32$, $float64$, $uint8$ und so weiter sein.
\end{itemize}
\cite{chollet2017dlpython}

Viele in dieser Arbeit verwendeten Tensoren beschreiben RBG-Bilddaten. Sie sind deshalb dritter Stufe, und die enthaltenen Daten sind Ganzzahlen zwischen $0$ und $255$. Die Form der durch das künstliche neuronale Netz generierten Bilder ist $(256, 256, 3)$. Die Dimensionen beschreiben hier also die Breite, Höhe und Farbtiefe der Bilddateien.


\section{Logistische Regression}
\label{sec:logreg}
Ein vergleichsweise einfacher Algorithmus des Maschinellen Lernens ist Logistic Regression oder Softmax Regression. Es lassen sich damit Klassifizierungen durchführen. Bei binären Klassifizierungen werden die Eingaben in zwei Kategorien unterteilt. Häufige Beispiele sind fehlerfreie oder fehlerhafte Produktionsergebnisse, ärztliche Befunde einer bestimmten Krankheit oder Gesundheit und ob in einem Bild ein bestimmtes Objekt vorhanden ist oder nicht.

Bei der Multiclass Logistic Regression werden die Eingaben in mehr als zwei Klassen unterteilt. Es lassen sich damit beispielsweise in einem Bild  verschiedene Arten von Fahrzeugen, Gegenständen oder Tieren unterteilen.

Diagramm: Ein-Schicht-NN (Die Ausgabeschicht wird nicht mitgezählt) Logistic Regeression kann mit einer einzelnen oder mehreren neuronalen Schichten imlementiert sein. Das künstliche neuronale Netz in Abb. x besitzt eine einzelne aus drei Einheiten bestehende Schicht. Eine Einheit besteht zunächst aus einer differenzierbaren Funktion. Oft ist das eine Multiplikation der Eingaben $X$ mit einer Anzahl lernbaren Parametern $W$ und eine anschließende Addition mit Bias-Werten $b$. Die Ausgabe einer Schicht wird oft mit $\hat{y}$ bezeichnet:
\begin{align}
\hat{y} = WX+b
\end{align}

Dieser folgt eine Aktivierungsfunktion, die dem errechneten Wert einen anderen Wert zuordnet, der entweder nahe $0$ oder nahe $1$ liegt (gelegentlich nahe $-1$ oder nahe $1$). Beispiele für Aktivierungsfunktionen sind Sigmoid:
\begin{align}
sig(t) = \frac{1}{1+e^{-t}}
\end{align}

und Rectified Linear Unit, kurz ReLU:
\begin{align}
f(x) = max(0, x)
\end{align}

oder auch Leaky ReLU:
\begin{align}
f(x) = max(0.01x, x)
\end{align}

In einigen künstlichen neuronalen Netzen werden zusätzlich die Eingaben für jede Schicht normalisiert, wodurch die Minimierung der Verlustfunktion (Gradient Descent) optimiert werden kann.

\section{Deep Neural Networks}
\label{sec:dnn}

\section{Convolutional Neural Networks}
\label{sec:cnn}
Für Bilddaten wendet ein CNN diese Operation typischerweise auf zwei dreidimensionale Matrizen an, nämlich einerseits auf die Eingabedaten und andererseits auf einen sogenannten Filter oder auch Kernel. Die Eingabedaten sind in der ersten Schicht des Netzes die RGB-Pixelinformationen und in allen weiteren konvolutionalen Schichten die Ausgabe der vorherigen Schicht. Ein Filter ist eine Anzahl von trainierbaren Parametern, in diesem Fall auch eine . Beide Matrizen haben also die Form $(H\ x\ W\ x\ C)$, wobei $H$ die Höhe, $W$ die Breite und $C$ die RGB-Farbwerte repräsentiert.

Jede der drei Matrixdimensionen variiert üblicherweise zwischen den verschiedenen Schichten des Netzes. In fast allen CNNs (\cite{Goodfellow-et-al-2016}, \cite{Lecun99objectrecognition}, \cite{RFB15a}, \cite{isola2018imagetoimage}) nimmt die Kardinalität zunächst ab. Die Reduktion kann durch die Konvolution selbst entstehen oder durch Pooling-Schichten. Beim Pooling werden aus benachbarten Matrixkoeffizienten meist das Maximum, seltener der Durchschnitt oder andere Aggregierungen gebildet. Auf diese Weise wird das neuronale Netz darauf trainiert die relevanten Informationen zu extrahieren. \cite{Goodfellow-et-al-2016}

Den konvolutionalen Schichten folgt in einigen Anwendungsfällen eine voll vernetzte Schicht (engl. Fully Connected Layer, FC), in der für jedes Neuron mit jedem Neuron der vorherigen Schicht eine Verbindung besteht. Besonders für die Bilderkennung ist diese Architektur gut geeignet. Die Ausgabe des neuronalen Netzes ist dann ein Vektor, beispielsweise von Wahrscheinlichkeitswerten für das Vorhandensein bestimmter Objekte und gegebenenfalls Bildkoordinaten der erkannten Objekte. Im Fall der Bildgenerierung ist die Ausgabe aber wieder eine Matrix von RGB-Pixelinformationen in der Form $(H\ x\ W\ x\ 3)$.

\section{U-Net-Architektur}
\label{sec:unet}
Die bis hierhin beschriebenen neuronalen Netze besitzen eine gradlinige Struktur, in der die Ausgabe einer Schicht nur an die nächste Schicht übergeben wird. Bei zunehmender Anzahl der Schichten verbessert sich die Performance neuronaler Netze mit diesem Aufbau zunächst, aber verschlechtert sich bei zu vielen Schichten wieder. In einem Residual Neural Network (ResNet) verhindern zusätzliche Verbindungen zwischen nicht direkt aufeinanderfolgenden Schichten diesen Performanceverlust.

Ein U-Net ist eine spezielle Form eines ResNets. Es hat eine annähernd symmetrische Struktur, in der sich die Kardinalitäten der Matrizen zuerst verringern und anschließend wieder erhöhen. U-Nets erzielen selbst mit wenigen Trainingsdaten gute Ergebnisse und benötigen dafür vergleichsweise wenig Rechenleistung. \cite{he2015deep}

\section{Generative Adversarial Networks}
\label{gan}
Ein Generative Adversarial Network (GAN)besteht zunächst aus einem Generator und einem Discriminator \cite{goodfellow2014generative}. Der Generator lernt während des
Trainings täuschend echt aussehende Bilddaten zu generieren. Der Discriminator
wird dagegen darauf trainiert, echte von generierte Bildern zu unterscheiden.
Anschließend können beide Modelle ``gegeneinander antreten''. Deswegen wird es
Generative \textit{Adversarial} Network genannt.

GANs unterscheiden sich von anderen Modellen durch ihren Aufbau und in Bezug auf das Trainingsziel. Künstliche neuronale Netze ermitteln häufig einen skalaren Wert wie beispielsweise einen Wahrscheinlichkeitswert und minimieren zu diesem Zweck eine Verlustfunktion. In einem GAN sind zwei CNNs im Einsatz. Das erste, der Generator, erstellt Tensoren n-ter Klasse. In diesem Beispiel sind das RGB-Bildinformationen. Das zweite CNN wird Discriminator genannt und bekommt als Eingabe die Ausgabe des ersten CNNs. Der Discriminator wird darauf trainiert, generierte Bilder von Bildern aus dem Trainingsset zu unterscheiden. Er minimiert also eine Verlustfunktion. Der Generator wird darauf trainiert, diese Verlustfunktion zu maximieren. Dieser Vorgang stammt aus der Spieltheorie und heißt Minimax. \cite{goodfellow2014generative}

Eine Erweiterung des GANs ist das Conditional Generative Adversarial Net (cGAN). In cGANs erhält der Generator zusätzliche Eingabedaten ($y$, ``Ground Truth''), die Hinweise für die Generierung enthalten. Der Discriminator erhält diese zusätzlichen Daten ebenfalls, um die Erkennung während des Trainings zu optimieren. \cite{mirza2014conditional}

\section{Image-To-Image-Translation}
\label{sec:pix2pix}
Der Image-To-Image-Translation-Algorithmus oder kurz Pix2Pix-Algorithmus verwendet ein GAN, um Bilder in Bilder zu übersetzen. Dafür kommen zwei CNNs zum Einsatz, nämlich eins für den Generator und eins für den Discriminator. \cite{isola2018imagetoimage}

Die Datei \hyperref[pix2pixpy]{main.py} befindet sich im Anhang und ist die Implementierung, die für diese Arbeit hauptsächlich verwendet wurde.

Die erste Funktion namens \lstinline|load| öffnet eine Datei und liest diese als JPEG-Bilddatei. Sie wird verwendet, um die Trainingsdaten zu laden. Weil die Trainingsdaten aus zwei Bildern pro Datei bestehen, nämlich jeweils einem Eingabebild und dem erwarteten Ergebnis (Ground Truth), extrahiert die Funktion zusätzlich die beiden Bilder und gibt jede in einer eigenen Variablen zurück.

Die Funktion \lstinline|resize| ist ebenfalls für die Verarbeitung zweier Bilder vorgesehen und skaliert beide auf die übergebene Breite und Höhe.

Die Funktion \lstinline|random_crop| stellt einen zufälligen Ausschnitt der zwei Eingabebilder frei. Sie wird aufgerufen, nachdem die Eingabebilder hochskaliert wurden. In der Referenzimplementierung wie auch in dieser Arbeit beträgt die Bildgröße vorher $286x286$ Pixel und $256x256$ Pixel nach dem Freistellen. Die zurückgegebenen Bilder haben dann wieder die gleichen Abmessungen wie die Eingabedaten.

In der Funktion \lstinline|normalize| werden die RGB-Werte der beiden Eingabebilder normalisiert. Dadurch sollen zu große Schritte auf dem Gradienten verhindert werden, die das Konvergieren verhindern könnten. \cite{chollet2017dlpython} Die RGB-Werte sind zunächst als Ganzzahlen im Bereich von 0 bis 255 gegeben und werden in Fließkommazahlen im Bereich -1 bis 1 umgerechnet.

In \lstinline|random_jitter| wird zunächst \lstinline|resize| aufgerufen, um die Eingabebilder auf 286x286 Pixel zu skalieren. Anschließend wird \lstinline|random_crop| auf die skalierten Bilder angewendet. Schließlich werden aufgrund einer Zufallszahl beide Eingabebilder mit einer Wahrscheinlichkeit von 50 Prozent gespiegelt.

Die beiden Funktionen \lstinline|load_image_train| und \lstinline|load_image_test| laden die Eingabe- und Referenzbilder (Ground Truth), skalieren diese auf 256x256 Pixel und normalisieren die Eingabedaten durch Aufrufe der Funktionen \lstinline|load|, \lstinline|resize| und \lstinline|normalize|. Nur die Funktion zum Laden der Trainingsbilder ruft für die geladenen Bilder \lstinline|random_jitter| auf.

Die Funktion \lstinline|downsample| erzeugt eine Schicht eines CNNs mit optionaler Batch Normalization und der Leaky-ReLU-Aktivierungsfunktion. Diese Schichten werden im Encoder des Generators sowie im Discriminator verwendet. Batch Normalization kommt nur in der jeweils ersten Schicht des Generators und des Diskriminators nicht zum Einsatz.

In der Funktion \lstinline|upsample| entstehen die übrigen CNN-Schichten des Generators. Der Entfaltung, bei der die Faltung der Funktion \lstinline|downsample| umgekehrt wird (engl. Transposed Convolution \cite{zaccone2018tensorflow}), folgt hier immer die Batch Normalization sowie optional Dropout mit einem Wahrscheinlichkeitswert von $0.5$. Das ist in den ersten drei Schichten des Decoders im Generator der Fall und bedeutet, dass statistisch die Hälfte der Aktivierungen „fallengelassen“ werden, also nicht in das Trainingsergebnis eingehen. Als Aktivierungsfunktion kommt ReLU zum Einsatz.

In \lstinline|build_generator| werden die Schichten des Generators zusammengefügt. Im Encoder wird acht Mal \lstinline|downsample| aufgerufen, bis die Dimensionen des Eingabetensors, die am Anfang der Bildgröße entsprechen (Breite und Höhe, also $256x256$ Pixel), durch die Konvolution auf $1$ reduziert sind. Die Dimension, welche die Anzahl Farbkanäle der Eingabebilder repräsentiert, erhöht sich im Encoder auf den Wert $512$. Die Werte dieser Achse werden als Features \cite{zhang2020dive} \cite{chollet2017dlpython} \cite{zaccone2018tensorflow} bezeichnet.

Der Decoder ruft seinerseits sieben Mal \lstinline|upsample| auf und führt eine weitere Entfaltung durch, wodurch die Dimensionen des bearbeiteten Tensors wieder jeweils dieselbe Form wie in den Eingabebildern annehmen. Schließlich werden die Skip-Connections zwischen den Schichten 1 bis 8 des Encoders und des Decoders hergestellt.

In der Funktion \lstinline|generator_loss| ist die Backpropagation des Generators implementiert. Zuerst wird dafür die Kreuzentropie des Ergebnisses des Discriminators und eines gleich großen Tensors, der mit Einsen initialisiert wird, gebildet. TODO: crossentropy Das Ergebnis ist das GAN-Loss. Anschließend kommt die L1 Loss Function, auch Mean Absolute Error genannt, zum Einsatz:
\begin{align}
  \mathcal{L} _{L1}(G) = \mathbb{E}_{x,y,z}[\|y - G(x,z)\|_1]
\end{align}
wobei $x$ für die Eingabebilder, $y$ für die Ground Truth und $z$ für das Ergebnis des des Generators, also das generierte Bild, steht \cite{isola2018imagetoimage}. Dieses L1-Loss wird aus dem Durchschnitt der absoluten Differenz der Ground Truth und des Generatorergebnisses berechnet. Der Gesamtverlust des Generators ist das GAN-Loss plus das Produkt aus dem Regularisierungsparameter Lambda, der konstant $100$ beträgt, und dem L1-Loss. Die Funktion \lstinline|generator_loss| gibt den Gesamtverlust, das GAN-Loss und das L1-Loss zurück.

In \lstinline|build_discriminator| wird der Discriminator aus verschiedenen neuronalen Schichten zusammengesetzt. Zwei Tensoren \lstinline|input_image| und \lstinline|target_image| mit denselben Anzahlen Achsen und denselben Dimensionen wie die Eingabebilder werden zu einem einzelnen Tensor konkateniert. Anschließend wird für diesen Tensor dreimal \lstinline|downsample| aufgerufen, bis der Tensor die Dimensionen $(32x32x256)$ hat. Es folgt ein Padding mit Nullen, eine weitere Konvolution sowie Batch Normalization und die Leaky ReLU Aktivierungsfunktion. Nach einem weiteren Null-Padding und einer weiteren Konvolution hat der Tensor die Dimensionen $(30x30x1)$.

In der Funktion \lstinline|discriminator_loss| ist die Backpropagation des Discriminators implementiert. Aus jeweils einem Tensor mit den gleichen Dimensionen wie die Eingabebilder und mit Einsen initialisiert wird die Kreuzentropie mit einem Trainingsbild und dem Ergebnis des Generators gebildet. Die jeweiligen Werte für den Verlust des Discriminators werden als Gesamtverlust zurückgegeben.



Im Ausführungsstrang wird zunächst eine Option des GPU-Speichermanagements gewählt, um die Größe des reservierten Speichers dynamisch anwachsen zu lassen. \cite{zaccone2018tensorflow} Zu Beginn der Experimente hat sich diese Einstellung als vorteilhaft herausgestellt.


\chapter{Implementierung}

\section{Entwicklungsumgebung}
\label{sec:env}

\subsection{Ubuntu Linux}
\label{subsec:ubuntu}

\subsection{Python}
\label{subsec:python}

\subsection{Tensorflow}
\label{subsec:tensorflow}

\subsection{CUDA}
\label{subsec:cuda}
Cuda ist eine NVIDIA-proprietäre Hardware- und Software-Architektur.

Es ist das Schema, nach dem NVIDIA-Grafikkarten gebaut wurden, die sowohl traditionelle Grafik-Rendering-Aufgaben als auch allgemeine Aufgaben durchführen können. Zum Programmieren der CUDA GPUs wird die Sprache CUDA C verwendet. CUDA C ist im Wesentlichen die Programmiersprache C mit einer Handvoll Erweiterungen, welche die Programmierung hoch parallelisierter Maschinen wie NVIDIA GPUs ermöglichen. \cite{sanders2010cuda}

Anders als frühere GPU-Generationen, die Rechenressourcen in Vertex- und Pixelshader aufteilten, enthält die CUDA-Architektur eine einheitliche Shader-Pipeline, welche die Zuordnung allgemeiner Berechnungen zu jeder arithmetisch-logischen Einheit (ALU) auf dem Chip durch ein Programm erlaubt. Diese ALUs wurden mit einem Befehlssatz entworfen, der für allgemeine Berechnungen statt für spezielle Grafikberechnungen zugeschnitten ist. Weiterhin wurde den Execution Units auf der GPU freier Lese- und Schreibzugriff auf den Speicher sowie Zugriff auf einen softwaregesteuerten Cache, genannt Shared Memory, gegeben. \cite{sanders2010cuda}

Zusätzlich zu der Sprache für das Schreiben von Code für die GPU stellt NVIDIA einen spezialisierten Hardwaretreiber zur Verfügung, der die hohe Rechenleistung der CUDA-Architektur ausschöpft. Kenntnis der OpenGL- oder DirectX-Programmierschnittstellen ist nicht länger erforderlich. \cite{sanders2010cuda}

\subsection{Image-To-Image-Translation in Python}
Es besteht eine Auswahl an Beispielimplementierungen für Image-To-Image-Translation. Das Original-Paper verweist auf ein GitHub-Repository, das eine Lua-Implementierung zur Verfügung stellt. Wegen der besseren Eignung für Experimente auf einem lokalen Rechner durch CUDA-Unterstützung mit Tensorflow wird in dieser Arbeit Python verwendet.

Die verwendete Beispielimplementierung ist zum Zeitpunkt der Erstellung dieses Dokumentes unter https://github.com/tensorflow/docs/blob/master/site/en/tutorials/generative/pix2pix.ipynb zu finden. Eine an die aktuelle Version von Tensorflow und die Anforderungen dieser Arbeit angepasste Version befindet sich im Anhang.

Die wichtigste Änderung ist das selbst erstellte Dataset, das statt der im Beispiel verwendeten Gebäudefassaden geladen wird. Es wurden Ausgaben entfernt, wegen derer die Ausführung des Skriptes unterbrochen wurde. Dabei wurden Bildinformationen und Beispielbilder nach dem Laden des Datasets, dem Aufteilen in Eingabebild und Ground Truth, der Erzeugung von Bildrauschen, Down- und Upsampling sowie den ersten Trainingsdurchläufen ausgegeben, die während des Trainings der Skizzen- und Renderbilder nicht betrachtet werden mussten. Außerdem wurden bei jedem Training Diagramme des Generators und des Discriminators erzeugt.


\chapter{Experimente und Resultate}
\label{ch:conduct}

\section{Vorbereitung der Eingabedaten}
\label{sec:preparation}
Die Trainingsdaten liegen für das ``Quick, Draw!''-Dataset im NDJSON-Format und als Blender-Dateien beziehungsweise
Wavefront OBJ-Dateien vor. Die effizienteste Möglichkeit, die Skizzen und 3D-Modelle für die Verarbeitung in einem CNN vorzubereiten, ist das Rendern und  Speichern als Bilddateien. Als Dateiformat kommen JPEG oder PNG infrage. Beide Formate können leicht als Trainingsset mit Tensorflow geladen werden.

NDJSON steht für Newline Delimited JavaScript Object Notation. In einer solchen Datei sind also zeilenweise JSON-Objekte gespeichert. Für jede Zeichnung sind Angaben zu Motiv, Ort und Zeit enthalten. Außerdem ist angegeben, ob die künstliche Intelligenz des Minispiels das Motiv in der Zeichnung korrekt klassifiziert, also erkannt hat. Jede Zeichnung hat weiterhin eine eindeutige ID.

Der relevanteste Teil ist die ``Drawings''-Eigenschaft der JSON-Objekte, ein mehrdimensionales Array mit Bildkoordinaten. Es enthält mindestens X-Koordinaten und Y-Koordinaten in jeweils einem Array im ``Drawings''-Array. Indem Linien zwischen den Bildkoordinaten in der Reihenfolge der Arrayelemente in ein Bild gezeichnet und in einer Bilddatei gespeichert werden, können die Zeichnungen in beliebigen Bilddateiformaten gespeichert werden. Für diese Arbeit wurde die Konvertierung ebenfalls in Python realisiert.

Bei der Verarbeitung in einem Convolutional Neural Network spielt die Bildgröße
in Bezug auf die Verarbeitungszeit eine wichtige Rolle. Bildformate der Größe 256x256 oder kleiner sind üblich und gut geeignet. Für ein GAN ist es zwar nicht erforderlich, aber sinnvoll, für Ein- und Ausgabedaten dieselben Dimensionen festzulegen. Bei der Bildgenerierung aus Skizzen unterscheiden sich Ein- und Ausgabdedaten natürlich bei die Farbtiefe. Während die Skizzen Graustufenbilder sind, besitzen die generierten Bilder drei Farbkanäle (RGB).

Sowohl während der Entwicklung als auch zur Laufzeit kann es vorteilhaft sein, Farbwerte statt auf der oft verwendeten Skala von $0$ bis $255$ als Fließkommazahlen im Bereich $0,0$ bis $1,0$ darzustellen. Auch für die Ausgaben der einzelnen Schichten eines künstlichen neuronalen Netzes kann diese Normalisierung durchgeführt werden (TODO: BatchNorm).

Ein ebenso wichtiger wie aufwendiger Vorgang ist die Klassifizierung der Trainingsdaten, also die Zuweisung von Eingaben zu den erlernbaren Ergebnissen. Aufgrund der selbst erstellten Ausgabebilder existieren für diesen Zweck keine vorgefertigten Datasets. Die Sortierung und Zuweisung erfolgt deshalb manuell.


\section{Anwendung herkömmlicher Shader}
\label{sec:shader}
Shader definieren die Interaktion des Lichts mit der Oberfläche des Objekts. Dabei können Shader aus einem oder mehreren BSDFs (Bidirectional Scattering Distribution Function) bestehen, die wiederum von Mix-und Add-Shadern in der Zusammensetzung gemischt werden. \cite{wartmann2014blender}

In der Computergrafik wird die Darstellung der Oberfläche eines Objekts durch die drei Faktoren Material, Textur und Ausleuchtung bestimmt. Material ist die Grundfarbe der Oberfläche. Textur sind die physischen Charakteristiken der Oberfläche, und Ausleuchtung ist die Hintergrundbeleuchtung oder Licht, welches von Lichtern (Lampen) emittiert wird. \cite{blain2020blender}

In der Computergrafik ist ein Material die Farbe eines Objektes. Es legt fest, wie das sichtbare Spektrum des Lichts von der Oberfläche des Objekts reflektiert wird. Ein Material legt außerdem fest, ob die Oberfläche matt oder metallisch erscheint. \cite{blain2020blender}

Material (Farbe) kann entsprechend der drei Farbschemas RGB, HSV oder Hex dargestellt werden. Wie die Farbe in jedem Schema erscheint ist auch von einem Alpha-Wert, welcher für die Menge der Transparenz steht, abhängig. \cite{blain2020blender}

Texturen definieren das physische Erscheinungsbild einer Oberfläche, also etwa wie glatt oder uneben diese erscheint, oder ihre Struktur, welche die visuelle Wahrnehumg der physischen Beschaffenheit des Objekts definiert. Diese Definition bestimmt, woraus die Oberfläche besteht, wie Holz, Ziegelsteine, Wasser und so weiter. \cite{blain2020blender}

Texturen werden durch Algorithmen generiert, wie sie in Blender integriert sind (prozedurale Texturen) oder aus Bilddateien (Bildtexturen). \cite{blain2020blender}

Die Qualität eines Bildes hängts direkt von der Effektivität des Shading-Algorithmus ab, der wiederum von der Modellierungsmethode des Objektes abhängt. Zwei wesentliche Methoden der Objektbeschreibung werden häufig verwendet, nämlich Oberflächendefinition mittels mathematischer Gleichungen und Oberflächenapproximation durch Mosaike aus polygonalen Flächen. \cite{phong1975shading}

Polygonobjekte sind anfangs immer ungeglättet, das heißt, dass beim Rendern oder in der schattierten Ansicht zunächst immer die einzelnen Flächen zu sehen sind, aus denen sich das Objekt zusammensetzt. Eine mögliche, wenn auch in Sachen Renderzeit und Speicherverbrauch ungünstige Methode wäre, einfach das Objekt so weit in kleinere Flächen zu unterteilen, dass beim Rendern eine Fläche pro Pixel gerendert wird. In der praktischen Arbeit verwendet man deshalb einen Trick, bei dem die Übergänge zwischen den einzelnen Flächen ``glattgerechnet'' werden. Übliche Verfahren hier sind das Gouraud oder Phong Shading. \cite{wartmann2014blender}

\section{Hyperparameter}
\label{sec:hyperparams}
Die Pix2Pix-Referenzimplementierung ist bereits für die Übersetzung von Skizzen in Fotos eingestellt. Für das Training waren anfangs mehrere Tausend Epochs, also Trainingsdurchläufe, erfoderlich, um zufriedenstellende Ergebnisse zu sehen. TODO: Diese Zahl konnte durch ... verringert werden.

Die Eingabebilder sind 256x256 Pixel groß und besitzen einen Farbkanal für Graustufen. Sie werden am Anfang des Trainingsprozesses durch sogenanntes Jittering augmentiert. Dabei werden die Bilder zuerst auf 286x268 Pixel vergrößert und anschließend auf einen zufälligen 256x256 Pixel großer Ausschnitt wieder verkleinert. Diese  Pixelgrößen können im Experiment geändert werden.

Der Adam Optimierer \cite{kingma2017adam} erhält für die Learning-Rate den Wert $0,0002$. Das Momentum ist auf $0,9$ voreingestellt. Diese beiden Werte beinflussen die Lerngeschwindigkeit und sind in begrenztem Maße anpassbar.

\section{Performancebeobachtungen}
\label{sec:performance}


\chapter{Diskussion}
\label{sec:conclusion}
