\chapter{Einleitung}
\label{ch:einleitung}
Obwohl die Idee für eine Maschine, die anhand eingegebener Daten selbständig
Entscheidungen treffen kann und die ersten praktischen Ansätze für künstliche
neuronale Netze schon einige Jahrzehnte alt sind, findet der Einsatz derartiger
Algorithmen erst seit einigen Jahren statt. Viele Erfindungen, die vor 30 bis 50
Jahren in Filmen und Serien Science Fiction darstellten, sind inzwischen nicht
nur Realität, sondern auch alltagstauglich. Zu den wichtigsten Beispielen zählen
verbale Schnittstellen an Computersystemen und auch Armbanduhren, autonome
Fahrzeuge etwa in Gestalt von Parkassistenten und verschiedene Verfahren zur
biometrischen Identitätsprüfung.

Andere rasche technologische Fortschritte aus der jüngeren Vergangenheit haben
nicht immer nur die Lebensqualität der beteiligten Personen erhöht, sondern
stellten durch Missbrauch gelegentlich sogar Gefahren
dar. So wie das Internet auch zur Verbreitung von Falschinformationen und
die sichere Verschlüsselung von gespeicherten Daten auch für Erpressungen genutzt
werden kann, ist die Generierung von täuschend echten Bildern unter Umständen
geeignet, persönlichen, finanziellen oder anders gearteten Schaden zu verursachen.

Weiterhin existieren bei der Auswahl der Trainingsdaten für künstliche neuronale
Netze rechtliche Grenzen. Bilddaten sind in mehr als ausreichenden Mengen
vorhanden, berücksichtigen aber zum Beispiel nicht immer das Recht am eigenen
Bild. Für diese Bachelorarbeit sind die Anforderungen an die Bildqualität außerdem sehr
hoch, da möglichst auch Texturen und Lichtreflexionen erlernt werden sollen.
Modellgrafiken mit geeigneten Material- und Beleuchtungseigenschaften
gibt es zwar auch, aber nur in weitaus geringeren Mengen. Für solche Fremdarbeiten,
die sehr arbeitsaufwendig sind, wäre auch die Klärung der Nutzungsrechte
erforderlich geworden. Für Trainingsergebnisse, die das mentale Modell einer
breiten Nutzergemeinschaft reflektieren, sind grundsätzlich auch Daten aus
möglichst vielen verschiedenen Quellen erforderlich.

Es ist deshalb eine Brücke geschlagen worden zwischen Trainingsdaten, die
zum einen aus zufällig ausgewählte Benutzereingaben bestehen, und solchen, die
bestimmte Qualitätseigenschaften erfüllen und in beliebiger Menge erstellt werden
können. Ein künstliches neuronales Netz soll aus Skizzen des
``Quick, Draw!''-Datasets von Google hochwertig gestaltete Figuren
generieren. TODO: Hierfür kommen als Ein- und Ausgabedaten sowohl Bilddateien als
auch die jeweils zugrundeliegenden Dateiformate NDJSON und Wavefront OBJ infrage.

\section{Ziel der Arbeit}
\label{sec:ziel}
Die Arbeit verfolgt das Ziel, verschiedene bewährte Architekturen künstlicher
neuronaler Netze zur Generierung von Bildern zu untersuchen und in einem
praxisorientierten Zusammenhang zu testen. Ich zeige mehrere Möglichkeiten,
wirklichkeitsnahe Bilder von Alltagsgegenständen aus Skizzen, die durch Benutzer
erstellt wurden, mittels eines künstlichen neuronalen Netzes zu generieren. Die
Bilddateien der Skizzen sowie die generierten Bilddateien können dabei aus
RBG-Pixelinformationen bestehen oder die Grafik mittels Bildkoordinaten beschreiben,
wie es bei Vektorgrafiken und 3D-Modellen der Fall ist. Eine Anwendung der
Ergebnisse ist beispielsweise als Feature eines Grafiktablets oder eines
Smartboards denkbar.

\section{Bisherige Arbeiten}
\label{sec:related}
Künstliche neuronale Netze finden erst seit wenigen Jahren breite Aufmerksamkeit,
seit auch Heimcomputer in der Lage sind die hohe Anzahl der erforderlichen
Rechenoperationen in annehmbarer Zeit auszuführen. Seitdem sind wenige,
englischsprachige Einführungen in die Thematik entstanden. Ein häufig genanntes
Buch ist ``Deep Learning'', das online kostenfrei zugänglich ist \cite{Goodfellow-et-al-2016}.
Ebenfalls online kostenfrei ist das Buch ``Dive into Deep Learning'' \cite{zhang2020dive}.
Ein weiteres, praxisorientierteres Buch ist ``Deep Learning with Python'' \cite{chollet2017}, dessen
zweite Auflage bald herausgegeben wird.

Als Architektur für ein künstliches neuronales Netz kommt für diese Arbeit
zunächst derselbe Aufbau wie in ``Image-To-Image-Translation'' \cite{isola2018imagetoimage} zum Einsatz. Dieser setzt seinerseits auf Generative
Adversarial Networks \cite{goodfellow2014generative} und Conditional Generative Adversarial Networks \cite{mirza2014conditional} auf.

In ``Unsupervised Representation Learning with Deep Convolutional Generative Adversarial Networks'' \cite{radford2016unsupervised} werden ebenfalls
realitätsnahe Ergebnisse erzielt. Anders als bei den bisher erwähnten Arbeiten
wird Unsupervised Learning verwendet, um das künstliche neuronale Netz zu trainieren.
