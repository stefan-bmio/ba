\chapter{Einleitung}
\label{ch:einleitung}

\section{Ziel der Arbeit}
\label{sec:ziel}
Die Arbeit verfolgt das Ziel, verschiedene bewährte Architekturen künstlicher
neuronaler Netze zur Generierung von Bildern zu untersuchen und in einem
praxisorientierten Zusammenhang zu testen. Ich zeige mehrere Möglichkeiten,
wirklichkeitsnahe Bilder von Alltagsgegenständen aus Skizzen, die durch Benutzer
erstellt wurden, mittels eines künstlichen neuronalen Netzes zu generieren. Die
Bilddateien der Skizzen sowie die generierten Bilddateien können dabei aus
RBG-Pixelinformationen bestehen oder die Grafik mittels Bildkoordinaten beschreiben,
wie es bei Vektorgrafiken und 3D-Modellen der Fall ist. Eine Anwendung der
Ergebnisse ist beispielsweise als Feature eines Grafiktablets oder eines
Smartboards denkbar.

\section{Bisherige Arbeiten}
\label{sec:related}
Künstliche neuronale Netze finden erst seit wenigen Jahren breite Aufmerksamkeit,
seit auch Heimcomputer in der Lage sind die hohe Anzahl der erforderlichen
Rechenoperationen in annehmbarer Zeit auszuführen. Seitdem sind wenige,
englischsprachige Einführungen in die Thematik entstanden. Ein häufig genanntes
Buch ist "Deep Learning", das online kostenfrei zugänglich ist \cite{Goodfellow-et-al-2016}.
Ebenfalls online kostenfrei ist das Buch "Dive into Deep Learning" \cite{zhang2020dive}.
Ein weiteres Buch ist
