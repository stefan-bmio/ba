%\KOMAoptions{DIV=15}
\KOMAoptions{DIV=22}
\titlehead{Berliner Hochschule für Technik Berlin\\Fachbereich VI -- Informatik und Medien}
%\subject{Bachelorarbeit}
\subject{Bachelorarbeit}
\title{Erzeugung von Bildern mittels\\neuronaler Netze}
\author{Stefan Berger\\Medieninformatik\\Matrikel-Nr. 854184}
\date{Berlin, 23. Juni 2022}

%Betreuerangaben
%\publishers{Betreut von Prof.~Dr.-Ing.~G.~Tsatsaronis und Dr.-Ing.~M.~Hofmann}
%bei extern betreute Arbeit:
\publishers{Betreut von: Prof.~Dr.~F.~Gers \\
\bigskip Gutachter: Prof.~Dr.~J.~Schimkat}

%\dedication{Widmung}

\maketitle

%\KOMAoptions{DIV=11}
\KOMAoptions{DIV=15}
\begin{abstract}
\vspace{\baselineskip}
Künstliche Intelligenz hat ein breites Anwendungsspektrum. Das Erkennen von Objekten in einem Bild, nachträgliche Manipulation von Bildern in Fotobearbeitungssoftware mithilfe von Filtern und Korrekturmechanismen oder Texterkennung sogar von Handschriften sind nur einige Beispiele für Ergebnisse, die maschinelles Lernen allein im visuellen Bereich erzielt.

Die Mathematik künstlicher neuronaler Netze hat bereits eine längere Geschichte. Trotz der verschiedenen Einsatzbereiche weisen die unterschiedlichen Algorithmen in ihrem Aufbau einige Ähnlichkeiten auf. In dieser Arbeit werde ich die Grundlagen des maschinellen Lernens erarbeiten. Anhand ausgewählter Implementierungen werde ich die Möglichkeit untersuchen, Bilder durch ein künstliches neuronales Netz generieren zu lassen.

Im Experiment und im Inhalt dieser Bachelorarbeit wird die Frage beantwortet, ob ein künstliches neuronales Netz lernen kann, Bilder zu generieren. Objekte auf generierten Bildern besitzen Eigenschaften, die durch einen Shader berechnet werden. Ein Shader ist ein Algorithmus, der für eine Oberfläche eine Textur und Lichtreflexionen berechnet. Kann ein künstliches neuronales Netz darauf trainiert werden, die Form, Textur und die Reflexionseigenschaften eines Objektes zu lernen?

\end{abstract}
