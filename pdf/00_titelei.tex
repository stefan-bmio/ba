%\KOMAoptions{DIV=15}
\KOMAoptions{DIV=22}
\titlehead{Berliner Hochschule für Technik Berlin\\Fachbereich VI -- Informatik und Medien}
%\subject{Bachelorarbeit}
\subject{Bachelorarbeit}
\title{Erzeugung von Bildern mittels Neuronalen Netzen}
\author{Stefan Berger\\Medieninformatik\\Matrikel-Nr. 854184}
\date{Berlin, 6. April 2021}

%Betreuerangaben
%\publishers{Betreut von Prof.~Dr.-Ing.~G.~Tsatsaronis und Dr.-Ing.~M.~Hofmann}
%bei extern betreute Arbeit:
\publishers{Betreut von: Prof.~Dr.~F.~Gers \\ \bigskip Gutachter: Prof.~Dr.~J.~Schimkat}

%\dedication{Widmung}

\maketitle

%\KOMAoptions{DIV=11}
\KOMAoptions{DIV=15}
\begin{abstract}
\vspace{\baselineskip}
Im Experiment und im Inhalt dieser Bachelorarbeit werden die folgenden Fragen beantwortet:

\noindent Kann ein künstliches neuronales Netz lernen Bilder zu generieren?

\noindent Ein Shader ist ein Programm, das eine Textur für ein Objekt berechnet, genau wie ein Programm das tut. Kann ein künstliches neuronales Netz darauf trainiert werden, die Textur eines Objektes zu lernen?
\end{abstract}
