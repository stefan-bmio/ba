\part*{Anhang}
\addcontentsline{toc}{chapter}{Anhang}

\chapter{Quelltexte}
\label{ch:a_sim}

\section{Image-To-Image-Translation main.py}
\label{pix2pixpy}

\begin{lstlisting}
  import tensorflow as tf

import os
import time
import datetime

from matplotlib import pyplot as plt
from IPython import display

from tensorflow.compat.v1 import ConfigProto
from tensorflow.compat.v1 import InteractiveSession


def load(image_file):
    # Read and decode an image file to a uint8 tensor
    image = tf.io.read_file(image_file)
    image = tf.image.decode_jpeg(image)

    # Split each image tensor into two tensors:
    w = tf.shape(image)[1]
    w = w // 2
    input_image = image[:, :w, :]
    real_image = image[:, w:, :]

    # Convert both images to float32 tensors
    input_image = tf.cast(input_image, tf.float32)
    real_image = tf.cast(real_image, tf.float32)

    return input_image, real_image


def resize(input_image, real_image, height, width):
    input_image = tf.image.resize(input_image, [height, width],
                                  method=tf.image.ResizeMethod.NEAREST_NEIGHBOR)
    real_image = tf.image.resize(real_image, [height, width],
                                 method=tf.image.ResizeMethod.NEAREST_NEIGHBOR)

    return input_image, real_image


def random_crop(input_image, real_image):
    stacked_image = tf.stack([input_image, real_image], axis=0)
    cropped_image = tf.image.random_crop(
        stacked_image, size=[2, IMG_HEIGHT, IMG_WIDTH, 3])

    return cropped_image[0], cropped_image[1]


# Normalizing the images to [-1, 1]
def normalize(input_image, real_image):
    input_image = (input_image / 127.5) - 1
    real_image = (real_image / 127.5) - 1

    return input_image, real_image


@tf.function()
def random_jitter(input_image, real_image):
    # Resizing to 286x286
    input_image, real_image = resize(input_image, real_image, 286, 286)

    # Random cropping back to 256x256
    input_image, real_image = random_crop(input_image, real_image)

    if tf.random.uniform(()) > 0.5:
        # Random mirroring
        input_image = tf.image.flip_left_right(input_image)
        real_image = tf.image.flip_left_right(real_image)

    return input_image, real_image


def load_image_train(image_file):
    input_image, real_image = load(image_file)
    input_image, real_image = random_jitter(input_image, real_image)
    input_image, real_image = normalize(input_image, real_image)

    return input_image, real_image


def load_image_test(image_file):
    input_image, real_image = load(image_file)
    input_image, real_image = resize(input_image, real_image,
                                     IMG_HEIGHT, IMG_WIDTH)
    input_image, real_image = normalize(input_image, real_image)

    return input_image, real_image


def downsample(filters, size, apply_batchnorm=True):
    initializer = tf.random_normal_initializer(0., 0.02)

    result = tf.keras.Sequential()
    result.add(
        tf.keras.layers.Conv2D(filters, size, strides=2, padding='same',
                               kernel_initializer=initializer, use_bias=False))

    if apply_batchnorm:
        result.add(tf.keras.layers.BatchNormalization())

    result.add(tf.keras.layers.LeakyReLU())

    return result


def upsample(filters, size, apply_dropout=False):
    initializer = tf.random_normal_initializer(0., 0.02)

    result = tf.keras.Sequential()
    result.add(
        tf.keras.layers.Conv2DTranspose(filters, size, strides=2,
                                        padding='same',
                                        kernel_initializer=initializer,
                                        use_bias=False))

    result.add(tf.keras.layers.BatchNormalization())

    if apply_dropout:
        result.add(tf.keras.layers.Dropout(0.5))

    result.add(tf.keras.layers.ReLU())

    return result


def build_generator():
    inputs = tf.keras.layers.Input(shape=[256, 256, 3])

    down_stack = [
        downsample(64, 4, apply_batchnorm=False),  # (batch_size, 128, 128, 64)
        downsample(128, 4),  # (batch_size, 64, 64, 128)
        downsample(256, 4),  # (batch_size, 32, 32, 256)
        downsample(512, 4),  # (batch_size, 16, 16, 512)
        downsample(512, 4),  # (batch_size, 8, 8, 512)
        downsample(512, 4),  # (batch_size, 4, 4, 512)
        downsample(512, 4),  # (batch_size, 2, 2, 512)
        downsample(512, 4),  # (batch_size, 1, 1, 512)
    ]

    up_stack = [
        upsample(512, 4, apply_dropout=True),  # (batch_size, 2, 2, 1024)
        upsample(512, 4, apply_dropout=True),  # (batch_size, 4, 4, 1024)
        upsample(512, 4, apply_dropout=True),  # (batch_size, 8, 8, 1024)
        upsample(512, 4),  # (batch_size, 16, 16, 1024)
        upsample(256, 4),  # (batch_size, 32, 32, 512)
        upsample(128, 4),  # (batch_size, 64, 64, 256)
        upsample(64, 4),  # (batch_size, 128, 128, 128)
    ]

    initializer = tf.random_normal_initializer(0., 0.02)
    last = tf.keras.layers.Conv2DTranspose(OUTPUT_CHANNELS, 4,
                                           strides=2,
                                           padding='same',
                                           kernel_initializer=initializer,
                                           activation='tanh')  # (batch_size, 256, 256, 3)

    x = inputs

    # Downsampling through the model
    skips = []
    for down in down_stack:
        x = down(x)
        skips.append(x)

    skips = reversed(skips[:-1])

    # Upsampling and establishing the skip connections
    for up, skip in zip(up_stack, skips):
        x = up(x)
        x = tf.keras.layers.Concatenate()([x, skip])

    x = last(x)

    return tf.keras.Model(inputs=inputs, outputs=x)


def generator_loss(disc_generated_output, gen_output, target):
    gan_loss = loss_object(tf.ones_like(disc_generated_output), disc_generated_output)

    # Mean absolute error
    l1_loss = tf.reduce_mean(tf.abs(target - gen_output))

    total_gen_loss = gan_loss + (LAMBDA * l1_loss)

    return total_gen_loss, gan_loss, l1_loss


def build_discriminator():
    initializer = tf.random_normal_initializer(0., 0.02)

    inp = tf.keras.layers.Input(shape=[256, 256, 3], name='input_image')
    tar = tf.keras.layers.Input(shape=[256, 256, 3], name='target_image')

    x = tf.keras.layers.concatenate([inp, tar])  # (batch_size, 256, 256, channels*2)

    down1 = downsample(64, 4, False)(x)  # (batch_size, 128, 128, 64)
    down2 = downsample(128, 4)(down1)  # (batch_size, 64, 64, 128)
    down3 = downsample(256, 4)(down2)  # (batch_size, 32, 32, 256)

    zero_pad1 = tf.keras.layers.ZeroPadding2D()(down3)  # (batch_size, 34, 34, 256)
    conv = tf.keras.layers.Conv2D(512, 4, strides=1,
                                  kernel_initializer=initializer,
                                  use_bias=False)(zero_pad1)  # (batch_size, 31, 31, 512)

    batchnorm1 = tf.keras.layers.BatchNormalization()(conv)

    leaky_relu = tf.keras.layers.LeakyReLU()(batchnorm1)

    zero_pad2 = tf.keras.layers.ZeroPadding2D()(leaky_relu)  # (batch_size, 33, 33, 512)

    last = tf.keras.layers.Conv2D(1, 4, strides=1,
                                  kernel_initializer=initializer)(zero_pad2)  # (batch_size, 30, 30, 1)

    return tf.keras.Model(inputs=[inp, tar], outputs=last)


def discriminator_loss(disc_real_output, disc_generated_output):
    real_loss = loss_object(tf.ones_like(disc_real_output), disc_real_output)

    generated_loss = loss_object(tf.zeros_like(disc_generated_output), disc_generated_output)

    total_disc_loss = real_loss + generated_loss

    return total_disc_loss


def generate_images(model, test_input, tar, image_index):
    prediction = model(test_input, training=True)
    plt.figure(figsize=(15, 15))

    display_list = [test_input[0], tar[0], prediction[0]]
    title = ['Input Image', 'Ground Truth', 'Predicted Image']

    for i in range(3):
        plt.subplot(1, 3, i + 1)
        plt.title(title[i])
        # Getting the pixel values in the [0, 1] range to plot.
        plt.imshow(display_list[i] * 0.5 + 0.5)
        plt.axis('off')
    plt.savefig('results/' + str(image_index) + '.jpg')


@tf.function
def train_step(input_image, target, step):
    with tf.GradientTape() as gen_tape, tf.GradientTape() as disc_tape:
        gen_output = generator(input_image, training=True)

        disc_real_output = discriminator([input_image, target], training=True)
        disc_generated_output = discriminator([input_image, gen_output], training=True)

        gen_total_loss, gen_gan_loss, gen_l1_loss = generator_loss(disc_generated_output, gen_output, target)
        disc_loss = discriminator_loss(disc_real_output, disc_generated_output)

    generator_gradients = gen_tape.gradient(gen_total_loss,
                                            generator.trainable_variables)
    discriminator_gradients = disc_tape.gradient(disc_loss,
                                                 discriminator.trainable_variables)

    generator_optimizer.apply_gradients(zip(generator_gradients,
                                            generator.trainable_variables))
    discriminator_optimizer.apply_gradients(zip(discriminator_gradients,
                                                discriminator.trainable_variables))

    with summary_writer.as_default():
        tf.summary.scalar('gen_total_loss', gen_total_loss, step=step // 1000)
        tf.summary.scalar('gen_gan_loss', gen_gan_loss, step=step // 1000)
        tf.summary.scalar('gen_l1_loss', gen_l1_loss, step=step // 1000)
        tf.summary.scalar('disc_loss', disc_loss, step=step // 1000)


def fit(train_ds, test_ds, steps):
    example_input, example_target = next(iter(test_ds.take(1)))
    start = time.time()

    for step, (input_image, target) in train_ds.repeat().take(steps).enumerate():
        if step % 1000 == 0:
            display.clear_output(wait=True)

            if step != 0:
                print(f'Time taken for 1000 steps: {time.time() - start:.2f} sec\n')

            start = time.time()

            generate_images(generator, example_input, example_target, step.numpy() // 1000)
            print(f"Step: {step // 1000}k")

        train_step(input_image, target, step)

        # Training step
        if (step + 1) % 10 == 0:
            print('.', end='', flush=True)

        # Save (checkpoint) the model every 5k steps
        if (step + 1) % 5000 == 0:
            checkpoint.save(file_prefix=checkpoint_prefix)


if __name__ == '__main__':
    config = ConfigProto()
    config.gpu_options.allow_growth = True
    session = InteractiveSession(config=config)

    # Adjust this value to the number of training images
    BUFFER_SIZE = 400
    # The batch size of 1 produced better results for the U-Net in the original pix2pix experiment
    BATCH_SIZE = 1
    # Each image is 256x256 in size
    IMG_WIDTH = 256
    IMG_HEIGHT = 256

    PATH = '../PIX2PIX/images/combined/candles/'
    train_dataset = tf.data.Dataset.list_files(PATH + 'train/*.png')
    train_dataset = train_dataset.map(load_image_train,
                                      num_parallel_calls=tf.data.AUTOTUNE)
    train_dataset = train_dataset.shuffle(BUFFER_SIZE)
    train_dataset = train_dataset.batch(BATCH_SIZE)

    try:
        test_dataset = tf.data.Dataset.list_files(str(PATH + 'test/*.png'))
    except tf.errors.InvalidArgumentError:
        test_dataset = tf.data.Dataset.list_files(str(PATH + 'val/*.png'))
    test_dataset = test_dataset.map(load_image_test)
    test_dataset = test_dataset.batch(BATCH_SIZE)

    OUTPUT_CHANNELS = 3

    generator = build_generator()

    LAMBDA = 100

    loss_object = tf.keras.losses.BinaryCrossentropy(from_logits=True)

    discriminator = build_discriminator()

    generator_optimizer = tf.keras.optimizers.Adam(2e-4, beta_1=0.5)
    discriminator_optimizer = tf.keras.optimizers.Adam(2e-4, beta_1=0.5)

    checkpoint_dir = './training_checkpoints'
    checkpoint_prefix = os.path.join(checkpoint_dir, "ckpt")
    checkpoint = tf.train.Checkpoint(generator_optimizer=generator_optimizer,
                                     discriminator_optimizer=discriminator_optimizer,
                                     generator=generator,
                                     discriminator=discriminator)

    log_dir = "logs/"

    summary_writer = tf.summary.create_file_writer(
        log_dir + "fit/" + datetime.datetime.now().strftime("%Y%m%d-%H%M%S"))

    fit(train_dataset, test_dataset, steps=40000)

    # Restoring the latest checkpoint in checkpoint_dir
    checkpoint.restore(tf.train.latest_checkpoint(checkpoint_dir))

    # Run the trained model on a few examples from the test set
    index = 1000
    for inp, tar in test_dataset.take(5):
        generate_images(generator, inp, tar, index)
        index = index + 1

\end{lstlisting}

\pagebreak

\section{Blender-Python-Script zur Generierung von Tischen}
\label{blenderpy}

\begin{lstlisting}
import bpy
import math
import random

obj_filepath = '/home/stefan/PycharmProjects/ba/blender/tables/obj/random_table{}.obj'
render_filepath = '/home/stefan/PycharmProjects/ba/blender/tables/rendered/random_table{}_render{}.jpg'

for tableIndex in range(0, 50):
    # add table top
    tableWidth = random.random() * 8 + 6
    tableHeight = random.random() * .5 + .1
    tableDepth = random.random() * 4 + 4
    legLength = random.random() * 4 + 1
    bpy.ops.mesh.primitive_cube_add(size=1, enter_editmode=False, location=(0, tableHeight / 2 + legLength, 0))
    bpy.context.object.name = 'Table'
    bpy.ops.transform.resize(value=(tableWidth, tableHeight, tableDepth), orient_type='GLOBAL', orient_matrix=((1, 0, 0), (0, 1, 0), (0, 0, 1)), orient_matrix_type='GLOBAL', constraint_axis=(False, True, False), mirror=True, use_proportional_edit=False, proportional_edit_falloff='SMOOTH', proportional_size=1, use_proportional_connected=False, use_proportional_projected=False)
    bpy.context.active_object.data.materials.append(bpy.data.materials.get("Wood"))

    # add table legs
    legWidth = random.random() * .5 + .1
    legDisplacement = random.random() * 1 + legWidth / 2
    legDisplacementX = tableWidth / 2 - legDisplacement
    legDisplacementZ = tableDepth / 2 - legDisplacement

    for legIndex, coords in enumerate([[legDisplacementX, legDisplacementZ], [-legDisplacementX, legDisplacementZ], [legDisplacementX, -legDisplacementZ], [-legDisplacementX, -legDisplacementZ]]):
        bpy.ops.mesh.primitive_cube_add(size=1, enter_editmode=False, location=(coords[0], legLength / 2, coords[1]))
        bpy.context.object.name = 'TableLeg' + str(legIndex)
        bpy.ops.transform.resize(value=(legWidth, legLength, legWidth), orient_type='GLOBAL', orient_matrix=((1, 0, 0), (0, 1, 0), (0, 0, 1)), orient_matrix_type='GLOBAL', constraint_axis=(False, True, False), mirror=True, use_proportional_edit=False, proportional_edit_falloff='SMOOTH', proportional_size=1, use_proportional_connected=False, use_proportional_projected=False)
        bpy.context.active_object.data.materials.append(bpy.data.materials.get("Wood"))

    # export to wavefront obj format
    bpy.ops.export_scene.obj(filepath = obj_filepath.format(tableIndex))

    # rotate and render
    bpy.ops.object.select_all(action='DESELECT')
    bpy.data.objects['Table'].select_set(True)
    for legIndex in range(0, 4):
        bpy.data.objects['TableLeg' + str(legIndex)].select_set(True)
    bpy.ops.object.transform_apply(location=False, rotation=True, scale=False)
    for renderIndex in range(0, 5):
        # rotate table
        bpy.ops.transform.rotate(value=math.pi / 5, orient_axis='Y', orient_type='GLOBAL', orient_matrix=((1, 0, 0), (0, 1, 0), (0, 0, 1)), orient_matrix_type='GLOBAL', constraint_axis=(False, True, False), mirror=True, use_proportional_edit=False, proportional_edit_falloff='SMOOTH', proportional_size=1, use_proportional_connected=False, use_proportional_projected=False)

        # render
        bpy.context.scene.render.filepath = render_filepath.format(tableIndex, renderIndex)
        bpy.ops.render.render(write_still = True)

    #delete objects
    bpy.ops.object.delete(use_global=False)
\end{lstlisting}
