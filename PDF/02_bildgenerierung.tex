\chapter{Modell}
\label{ch:model}


\section{Logistische Regression}
\label{sec:logreg}
Ein vergleichsweise einfacher Algorithmus des Maschinellen Lernens ist Logistic Regression oder Softmax Regression. Es lassen sich damit Klassifizierungen durchführen. Bei binären Klassifizierungen werden die Eingaben in zwei Kategorien unterteilt. Häufige Beispiele sind fehlerfreie oder fehlerhafte Produktionsergebnisse, ärztliche Befunde einer bestimmten Krankheit oder Gesundheit und ob in einem Bild ein bestimmtes Objekt vorhanden ist oder nicht.

Bei der Multiclass Logistic Regression werden die Eingaben in mehr als zwei Klassen unterteilt. Es lassen sich damit beispielsweise in einem Bild  verschiedene Arten von Fahrzeugen, Gegenständen oder Tieren unterteilen.

Diagramm: Ein-Schicht-NN (Die Ausgabeschicht wird nicht mitgezählt) Logistic Regeression kann mit einer einzelnen oder mehreren neuronalen Schichten imlementiert sein. Das künstliche neuronale Netz in Abb. x besitzt eine einzelne aus drei Einheiten bestehende Schicht. Eine Einheit besteht zunächst aus einer differenzierbaren Funktion. Oft ist das eine Multiplikation der Eingaben $X$ mit einer Anzahl lernbaren Parametern $W$ und eine anschließende Addition mit Bias-Werten $b$. Die Ausgabe einer Schicht wird oft mit $\hat{y}$ bezeichnet:
\begin{align}
\hat{y} = WX+b
\end{align}

Dieser folgt eine Aktivierungsfunktion, die dem errechneten Wert einen anderen Wert zuordnet, der entweder nahe $0$ oder nahe $1$ liegt (gelegentlich nahe $-1$ oder nahe $1$). Beispiele für Aktivierungsfunktionen sind Sigmoid:
\begin{align}
sig(t) = \frac{1}{1+e^{-t}}
\end{align}

und Rectified Linear Unit, kurz ReLU:
\begin{align}
f(x) = max(0, x)
\end{align}

oder auch Leaky ReLU:
\begin{align}
f(x) = max(0.01x, x)
\end{align}

In einigen künstlichen neuronalen Netzen werden zusätzlich die Eingaben für jede Schicht normalisiert, wodurch die Minimierung der Verlustfunktion (Gradient Descent) optimiert werden kann.

\section{Deep Neural Networks}
\label{sec:dnn}

\section{Convolutional Neural Networks}
\label{sec:cnn}
Für Bilddaten wendet ein CNN diese Operation typischerweise auf zwei dreidimensionale Matrizen an, nämlich einerseits auf die Eingabedaten und andererseits auf einen sogenannten Filter oder auch Kernel. Die Eingabedaten sind in der ersten Schicht des Netzes die RGB-Pixelinformationen und in allen weiteren konvolutionalen Schichten die Ausgabe der vorherigen Schicht. Ein Filter ist eine Anzahl von trainierbaren Parametern, in diesem Fall auch eine . Beide Matrizen haben also die Form $(H\ x\ W\ x\ C)$, wobei $H$ die Höhe, $W$ die Breite und $C$ die RGB-Farbwerte repräsentiert.

Jede der drei Matrixdimensionen variiert üblicherweise zwischen den verschiedenen Schichten des Netzes. In fast allen CNNs (\cite{Goodfellow-et-al-2016}, \cite{Lecun99objectrecognition}, \cite{RFB15a}, \cite{isola2018imagetoimage}) nimmt die Kardinalität zunächst ab. Die Reduktion kann durch die Konvolution selbst entstehen oder durch Pooling-Schichten. Beim Pooling werden aus benachbarten Matrixkoeffizienten meist das Maximum, seltener der Durchschnitt oder andere Aggregierungen gebildet. Auf diese Weise wird das neuronale Netz darauf trainiert die relevanten Informationen zu extrahieren. \cite{Goodfellow-et-al-2016}

Den konvolutionalen Schichten folgt in einigen Anwendungsfällen eine voll vernetzte Schicht (engl. Fully Connected Layer, FC), in der für jedes Neuron mit jedem Neuron der vorherigen Schicht eine Verbindung besteht. Besonders für die Bilderkennung ist diese Architektur gut geeignet. Die Ausgabe des neuronalen Netzes ist dann ein Vektor, beispielsweise von Wahrscheinlichkeitswerten für das Vorhandensein bestimmter Objekte und gegebenenfalls Bildkoordinaten der erkannten Objekte. Im Fall der Bildgenerierung ist die Ausgabe aber wieder eine Matrix von RGB-Pixelinformationen in der Form $(H\ x\ W\ x\ 3)$.

\section{U-Net-Architektur}
\label{sec:unet}
Die bis hierhin beschriebenen neuronalen Netze besitzen eine gradlinige Struktur, in der die Ausgabe einer Schicht nur an die nächste Schicht übergeben wird. Bei zunehmender Anzahl der Schichten verbessert sich die Performance neuronaler Netze mit diesem Aufbau zunächst, aber verschlechtert sich bei zu vielen Schichten wieder. In einem Residual Neural Network (ResNet) verhindern zusätzliche Verbindungen zwischen nicht direkt aufeinanderfolgenden Schichten diesen Performanceverlust.

Ein U-Net ist eine spezielle Form eines ResNets. Es hat eine annähernd symmetrische Struktur, in der sich die Kardinalitäten der Matrizen zuerst verringern und anschließend wieder erhöhen. U-Nets erzielen selbst mit wenigen Trainingsdaten gute Ergebnisse und benötigen dafür vergleichsweise wenig Rechenleistung. \cite{he2015deep}

\section{Generative Adversarial Networks}
\label{gan}
Ein Generative Adversarial Network (GAN)besteht zunächst aus einem Generator und einem Discriminator \cite{goodfellow2014generative}. Der Generator lernt während des
Trainings täuschend echt aussehende Bilddaten zu generieren. Der Discriminator
wird dagegen darauf trainiert, echte von generierte Bildern zu unterscheiden.
Anschließend können beide Modelle ``gegeneinander antreten''. Deswegen wird es
Generative \textit{Adversarial} Network genannt.

GANs unterscheiden sich von anderen Modellen durch ihren Aufbau und in Bezug auf das Trainingsziel. Künstliche neuronale Netze ermitteln häufig einen skalaren Wert wie beispielsweise einen Wahrscheinlichkeitswert und minimieren zu diesem Zweck eine Verlustfunktion. In einem GAN sind zwei CNNs im Einsatz. Das erste, der Generator, erstellt Tensoren n-ter Klasse. In diesem Beispiel sind das RGB-Bildinformationen. Das zweite CNN wird Discriminator genannt und bekommt als Eingabe die Ausgabe des ersten CNNs. Der Discriminator wird darauf trainiert, generierte Bilder von Bildern aus dem Trainingsset zu unterscheiden. Er minimiert also eine Verlustfunktion. Der Generator wird darauf trainiert, diese Verlustfunktion zu maximieren. Dieser Vorgang stammt aus der Spieltheorie und heißt Minimax. \cite{goodfellow2014generative}

Eine Erweiterung des GANs ist das Conditional Generative Adversarial Net (cGAN). In cGANs erhält der Generator zusätzliche Eingabedaten ($y$, ``Ground Truth''), die Hinweise für die Generierung enthalten. Der Discriminator erhält diese zusätzlichen Daten ebenfalls, um die Erkennung während des Trainings zu optimieren. \cite{mirza2014conditional}

\section{Image-To-Image-Translation}
\label{sec:pix2pix}
Der Image-To-Image-Translation-Algorithmus oder kurz Pix2Pix-Algorithmus verwendet
ein GAN, um Bilder in Bilder zu übersetzen. Dafür kommen zwei CNNs zum Einsatz, nämlich eins für den Generator und eins für den Discriminator.

\chapter{Implementierung}

\section{Entwicklungsumgebung}
\label{sec:env}

\subsection{Ubuntu Linux}
\label{subsec:ubuntu}

\subsection{Python}
\label{subsec:python}

\subsection{Tensorflow}
\label{subsec:tensorflow}

\subsection{CUDA}
\label{subsec:cuda}
Cuda ist eine NVIDIA-proprietäre Hardware- und Software-Architektur. 

Es ist das Schema, nach dem NVIDIA-Grafikkarten gebaut wurden, die sowohl traditionelle Grafik-Rendering-Aufgaben als auch allgemeine Aufgaben durchführen können. Zum Programmieren der CUDA GPUs wird die Sprache CUDA C verwendet. CUDA C ist im Wesentlichen die Programmiersprache C mit einer Handvoll Erweiterungen, welche die Programmierung hoch parallelisierter Maschinen wie NVIDIA GPUs ermöglichen. \cite{sanders2010cuda}

Anders als frühere GPU-Generationen, die Rechenressourcen in Vertex- und Pixelshader aufteilten, enthält die CUDA-Architektur eine einheitliche Shader-Pipeline, welche die Zuordnung allgemeiner Berechnungen zu jeder arithmetisch-logischen Einheit (ALU) auf dem Chip durch ein Programm erlaubt. Diese ALUs wurden mit einem Befehlssatz entworfen, der für allgemeine Berechnungen statt für spezielle Grafikberechnungen zugeschnitten ist. Weiterhin wurde den Execution Units auf der GPU freier Lese- und Schreibzugriff auf den Speicher sowie Zugriff auf einen softwaregesteuerten Cache, genannt Shared Memory, gegeben. \cite{sanders2010cuda}

Zusätzlich zu der Sprache für das Schreiben von Code für die GPU stellt NVIDIA einen spezialisierten Hardwaretreiber zur Verfügung, der die hohe Rechenleistung der CUDA-Architektur ausschöpft. Kenntnis der OpenGL- oder DirectX-Programmierschnittstellen ist nicht länger erforderlich. \cite{sanders2010cuda}


\chapter{Experimente und Resultate}
\label{ch:conduct}

\section{Vorbereitung der Eingabedaten}
\label{sec:preparation}
Die Trainingsdaten liegen im NDJSON-Format und als Blender-Dateien beziehungsweise
Wavefront OBJ-Dateien vor. Die effizienteste Möglichkeit, die Skizzen und 3D-Modelle für die Verarbeitung in einem CNN vorzubereiten, ist das Rendern und  Speichern als Bilddateien. Als Dateiformat kommen JPEG oder PNG infrage. Beide Formate können leicht als Trainingsset mit Tensorflow geladen werden.

Bei der Verarbeitung in einem Convolutional Neural Network spielt die Bildgröße
in Bezug auf die Verarbeitungszeit eine wichtige Rolle. Bildformate der Größe 256x256 oder kleiner sind üblich und gut geeignet. Für ein GAN ist es zwar nicht erforderlich, aber sinnvoll, für Ein- und Ausgabedaten dieselben Dimensionen festzulegen. Bei der Bildgenerierung aus Skizzen unterscheiden sich Ein- und Ausgabdedaten natürlich bei die Farbtiefe. Während die Skizzen Graustufenbilder sind, besitzen die generierten Bilder drei Farbkanäle (RGB).

Sowohl während der Entwicklung als auch zur Laufzeit kann es vorteilhaft sein, Farbwerte statt auf der oft verwendeten Skala von 0 bis 255 als Fließkommazahlen im Bereich 0,0 bis 1,0 darzustellen. Auch für die Ausgaben der einzelnen Schichten eines künstlichen neuronalen Netzes kann diese Normalisierung durchgeführt werden (TODO: BatchNorm).

Ein ebenso wichtiger wie aufwendiger Vorgang ist die Klassifizierung der Trainingsdaten, also die Zuweisung von Eingaben zu den erlernbaren Ergebnissen. Aufgrund der selbst erstellten Ausgabebilder existieren für diesen Zweck keine vorgefertigten Datasets. Die Sortierung und Zuweisung erfolgt deshalb manuell.



\section{Anwendung herkömmlicher Shader}
\label{sec:shader}

\section{Hyperparameter}
\label{sec:hyperparams}
Die Pix2Pix-Referenzimplementierung ist bereits für die Übersetzung von Skizzen in Fotos eingestellt. Für das Training waren anfangs mehrere Tausend Epochs, also Trainingsdurchläufe, erfoderlich, um zufriedenstellende Ergebnisse zu sehen. TODO: Diese Zahl konnte durch ... verringert werden.

Die Eingabebilder sind 256x256 Pixel groß und besitzen einen Farbkanal für Graustufen. Sie werden am Anfang des Trainingsprozesses durch sogenanntes Jittering augmentiert. Dabei werden die Bilder zuerst auf 286x268 Pixel vergrößert und anschließend auf einen zufälligen 256x256 Pixel großer Ausschnitt wieder verkleinert. Diese  Pixelgrößen können im Experiment geändert werden.

Der Adam Optimierer \cite{kingma2017adam} erhält für die Learning-Rate den Wert 0,0002. Das Momentum ist auf 0,9 voreingestellt. Diese beiden Werte beinflussen die Lerngeschwindigkeit und sind in begrenztem Maße anpassbar.

\section{Performancebeobachtungen}
\label{sec:performance}


\chapter{Diskussion}
\label{sec:conclusion}
